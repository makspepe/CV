%-------------------------------------------------------------------------------
%                             ADDITIONAL PACKAGES
%-------------------------------------------------------------------------------
% DO NOT MOVE THOSE PACKAGES OR IT WILL NOT WORK

\documentclass[
	a4paper,
	% 9pt,
	% sidesectionsize=Large,
	% showframes,
	% vline=2.2em,
	% maincolor=cvgreen,
	% sidecolor=gray!50,
	% sidetextcolor=green,
	% sectioncolor=red,
	% subsectioncolor=orange,
	% itemtextcolor=black!80,
	% sidebarwidth=0.4\paperwidth,
	% topbottommargin=0.03\paperheight,
	% leftrightmargin=20pt,
	% profilepicsize=4.5cm,
	% profilepicborderwidth=3.5pt,
	% profilepicstyle=profilecircle,
	% profilepiczoom=1.0,
	% profilepicxshift=0mm,
	% profilepicyshift=0mm,
	% profilepicrounding=1.0cm,
	% logowidth=4.5cm,
	% logospace=5mm,
	% logoposition=before,
	% sidebarplacement=right,
	% datecolwidth=0.22\textwidth,
]{killme}

% improve word spacing and hyphenation
\usepackage{microtype}
\usepackage{ragged2e}

% use a sans serif font as default
\usepackage[sfdefault]{ClearSans}
% \usepackage[sfdefault]{noto}

% take care of proper font encoding
\ifxetexorluatex
	\usepackage{fontspec}
	\defaultfontfeatures{Ligatures=TeX}
	% \newfontfamily\headingfont[Path=fonts/]{segoeuib.ttf} % use local font
\else
	\usepackage[utf8]{inputenc}
	\usepackage[T1]{fontenc}
\fi

% enable mathematical syntax for some symbols like \varnothing
\usepackage{amssymb}

% bubble diagram configuration
\usepackage{smartdiagram}
\smartdiagramset{
	% default font size is \large, so adjust to harmonize with sidebar layout
	bubble center node font = \footnotesize,
	bubble node font = \footnotesize,
	% default: 4cm/2.5cm; make minimum diameter relative to sidebar size
	bubble center node size = 0.4\sidebartextwidth,
	bubble node size = 0.25\sidebartextwidth,
	distance center/other bubbles = 1.5em,
	% set center bubble color
	bubble center node color = maincolor!70,
	% define the list of colors usable in the diagram
	set color list = {maincolor!10, maincolor!40,
	maincolor!20, maincolor!60, maincolor!35},
	% sets the opacity at which the bubbles are shown
	bubble fill opacity = 0.8,
}

%-------------------------------------------------------------------------------
%                            PERSONAL INFORMATION
%-------------------------------------------------------------------------------
%% mandatory information
% your name
\cvname{Пепеляев Максим}
% job title/career
\cvjobtitle{Младший разработчик,\\[0.2em] Студент НИУ ВШЭ}

%% optional information
% profile picture
\cvprofilepic{pics/profile.png}
% logo picture
%\cvlogopic{pics/logo_txt.png}

% NOTE: ordering in sidebar will mimic the following order
% date of birth
\cvbirthday{05.06.1999, \getage{05/06/1999} года} %%% TODO - make so that "года" is changed to a proper form automatically

% short address/location, use \newline if more than 1 line is required
\cvaddress{Россия, Пермь}{https://www.google.com/maps/place/Perm,+Perm+Krai/data=!4m2!3m1!1s0x43e8c6e1d886f20b:0x9b4aca02b87a8a0e?sa=X&ved=2ahUKEwi3qor_vYH9AhUPxosKHWe_AjEQ8gF6BAgdEAI}

% phone number
\cvphone{+7 996 082 72 94}

% personal website
%\cvsite{https://fagscience.net}

% email address
\cvmail{makspepe7@gmail.com}

% any other custom entry
%\cvcustomdata{\faFlag}{Uganda}

%-------------------------------------------------------------------------------
%                              SIDEBAR 1st PAGE
%-------------------------------------------------------------------------------
% colors to use in bubbles at the bottom of the page
\definecolor{pastelgreen}{HTML}{D7ECD9}
\definecolor{pastelpurple}{HTML}{D5D6EA}
\definecolor{pastelorange}{HTML}{F5D5CB}
\definecolor{pastelyellow}{HTML}{F6F6EB}

% add more profile sections to sidebar on first page
\addtofrontsidebar{
	\graphicspath{{pics/flags/}}

	% social network accounts incl. proper hyperlinks
	%\sidesection{Социальные Сети}
		\begin{icontable}{1.7em}{0.4em}
			\social{\faGithub}
				{https://github.com/makspepe}
				{makspepe}
		\end{icontable}
  		\begin{icontable}{1.7em}{0.4em} 
			\social{\faLinkedin}
				{https://www.linkedin.com/in/makspepe/}
				{makspepe}
		\end{icontable}
    	\begin{icontable}{1.7em}{0.4em}
			\social{\faPaperPlane }
				{https://t.me/maxpepe7}
				{maxpepe7}
		\end{icontable}

	\sidesection{Языки}
        \pointskill{\flag{United-Kingdom.png}}{Английский}{5}[6]
        \vspace{-1ex}\raggedleft{\color{maincolor}C1 Продвинутый}\par
		\pointskill{\flag{Germany.png}}{Немецкий}{3}[6]
        \vspace{-1ex}\raggedleft{\color{maincolor}B1 Средний}\par
		\pointskill{\flag{China.png}}{Китайский}{1}
        \vspace{-1ex}\raggedleft{\color{maincolor}А1 Начальный}\par

	\sidesection{Навыки} 
 		\pointskill{\faCode}{C\#}{3}[6]
          	\skill[1.6em]{\faCompress}{Unity}
 		\pointskill{\faCode}{C++}{2}[6]
            \skill[1.6em]{\faCompress}{UE}
 		\pointskill{\faCode}{Python}{2}[6]
             \skill[1.6em]{\faCompress}{numpy, scikit, pandas}
 		%\pointskill{\faCode}{Java}{1}[6]
        \pointskill{\faDatabase}{SQL}{2}[6]
            \skill[1.6em]{\faCompress}{Postgre, MySQL}
        \pointskill{\faColumns}{LaTeX}{2}[6]
        \skill{\faCodeBranch}{Git}
        
        \sidesection{Качества} 
        \raggedright{\chartlabel[pastelgreen]{Внимание к деталям}\chartlabel[pastelgreen]{Лидерство}\chartlabel[pastelorange]{Творческий взгляд}\chartlabel[pastelorange]{Адаптивность}\chartlabel[pastelpurple]{Решение проблем}} 
}

%-------------------------------------------------------------------------------
%                              SIDEBAR 2nd PAGE
%-------------------------------------------------------------------------------

\addtobacksidebar{
	
	% \begin{figure}\centering
	% 	\smartdiagram[bubble diagram]{
	% 		\textcolor{sidecolor}{\textbf{Being a}} \\
	% 		\textcolor{sidecolor}{\textbf{Retard}}, % center bubble
	% 		\textcolor{sidecolor!90}{Eating},
	% 		\textcolor{sidecolor!90}{Sleeping},
	% 		\textcolor{sidecolor!90}{Rolling},
	% 		\textcolor{sidecolor!90}{Playing},
	% 		\textcolor{sidecolor!90}{Chilling}
	% 	}
	% \end{figure}

	% \chartlabel{Wheel Chart}

	% \wheelchart{3.7em}{2em}{%
	% 20/3em/sidecolor!50/Chill,
	% 15/3em/sidecolor!15/Play,
	% 30/4em/sidecolor!40/Sleep,
	% 20/3em/sidecolor!20/Eat
	% }

	% \sidesection{Barskills}
 %    	\barskill[1ex]{\faSkyatlas}{Wearing asian rice hats}{60}
 %    	\barskill[2ex]{\faImage}{Playing Chess}{30}
 %    	\barskill[3ex]{\faMusic}{Playing the bamboo flute}{50}

	% \sidesection{Сертификаты}
	% \begin{memberships}
	% 	\membership[4em]{pics/stepik.png}{
 %            \href{https://stepik.org/cert/1908072}{Программирование на Python (Stepik)}
 %        }
 %        \membership[4em]{pics/stepik.png}{
 %            \href{https://stepik.org/cert/843179}{Введение в базы данных (Stepik)}
 %        }
	% 	\membership[4em]{pics/coursera.png}{
 %            \href{https://www.coursera.org/account/accomplishments/verify/DZYQD8E4QSJ3}{Управление человеческими ресурсами (Coursera)}
 %        }
	% \end{memberships}
}

\discardpages{2}% Discard these pages. Render only 1 page



%-------------------------------------------------------------------------------
%                         TABLE ENTRIES RIGHT COLUMN
%-------------------------------------------------------------------------------
\begin{document}
\makefrontsidebar

\cvsection{\faGraduationCap \enskip Образование}
\cvsubsection{Бакалавриат}
\begin{cvtable}[1.5]
	\cvitem{09/2019 -- 06/2023}{Программная инженерия GPA:8.0/10}{Высшая Школа Экономики (ВШЭ)}
	   {\par % без par он съедает первых параграф, только блять тут, я не знаю что сломано и мне похуй
        Основные темы: алгоритмы, управление программными проектами, проектирование и архитектура программного обеспечения, системы баз данных, машинное обучение, UX и написание документации.\par \vspace{0.1cm}
        Языки, используемые в практических заданиях и групповых проектах: C\#, C++, Assembler, Python, R, Java, JavaScript.\par 
        %Проходил обучение на факультативах по английскому языки и началам китайского языка. 
        %В факультатив по английскому входили устная и письменная речь в академическом контексте, перевод в профессиональном контексте. \par
        \vspace{0.15cm}} 
  	\cvitem{}{Выпускная квалификационная работа}{Высшая Школа Экономики (ВШЭ)}
		{Разработка инструмента для создания игровых уровней с использованием генеративных нейронных сетей.}
\end{cvtable}

\cvsubsection{Дополнительный профиль}
\begin{cvtable}[1.5]
	\cvitem{09/2020 -- 06/2022}{Майнор «Менеджмент» GPA:9.3/10}{Высшая Школа Экономики (ВШЭ)}
		{\par
        Основные темы: планирование, стратегический и операционный менеджмент, управление человеческими ресурсами и конфликтами, управление проектами, маркетинг.\par \vspace{0.1cm}
        Сформировал и руководил командой для опроса студентов, связи с преподавателями и корректировки траектории обучения в зависимости от результатов опросов.}
\end{cvtable}

\cvsection{\faBriefcase \enskip Мои проекты}
\begin{cvtable}
    \cvitem{12/2022 -- 06/2023}{Инструмент для создания игровых уровней с помощью генеративных нейронных сетей}{}{
    Обучение GAN на играх с похожим дизайном уровней, возможностью задавать параметры генерируемых уровней, модуль экспорта в редакторы UE и Unity.\vspace{0.15cm}}
    \cvitem{06/2022 -- 09/2022}{Разработка многопользовательского шутера | UE5}{}{
Система конпенсациии задержки сети на стороне клиента, настраиваемые режимы игры, редактор уровней.\vspace{0.15cm}}
    \cvitem{12/2021 -- 04/2022}{Разработка roguelike-игры | Unity}{}{
Роглайк шутер, процедурно генерируемые уровни, пазлы, инструменты для модификации.\vspace{0.15cm}}
	\cvitem{01/2020 -- 04/2020}{Нейронная сеть для автомобиля | Unity}{}{
 Глубинная нейронная сеть на C\#,  обучение с подкреплением соревнованием 20 машин на трассе в Unity.}

\end{cvtable}

\cvsection{\faBlackTie \enskip Опыт работы}
\begin{cvtable}
	\cvitem{05/2022 -- 06/2022}{Стажер в лаборатории ВШЭ}{Высшая Школа Экономики (ВШЭ)}{
Инструмент на HSE API: сбор данных с сайтов и API, форматирование информации, отправка данных в зависимости от аргументов запроса.}
\end{cvtable}

\cvsection{\faRocket \enskip Внеучебная деятельность}
\begin{cvtable}
    \cvitemshort{Моддинг}{Декомпиляция и модификация множества видео игр. Создание полноценных модов, инструментов, коллаб для: Mount \& Blade, Total War: Warhammer 2, Dawn of War, STALKER Anomaly, Rimworld, Slay the Spire.\vspace{0.15cm}} 
	\cvitemshort{Видео}{Создание видеороликов, творческие эксперименты с исследованием различных форм повествования.\vspace{0.15cm}}
	\cvitemshort{Музыка}{Начинающий аудиоинженер и композитор, 15+ композиций, в основном с использованием синтезатора и электрогитары.\vspace{0.15cm}}
	\cvitemshort{Обучение}{Написание углубленных руководств по использованию программных инструментов, моддингу, игровым стратегиям.}
\end{cvtable}

% \cvsection{Publications}
% \begin{cvtable}
% 	\cvpubitem{Cooking: 100 recipes for lazy Pandas}{Me and My Imaginary Friends}
% 		{Culinary World}{2030}
% \end{cvtable}

% \cvsection{Awards}
% \begin{cvtable}
% 	\cvitem{2019 -- now}{Retard of the Year}{Gaymer World Forum}{}
% 	\cvitem{2016 -- now}{Face of World Wide Fund for Retards}{WWF}{}
% \end{cvtable}

%-------------------------------------------------------------------------------
%                           SECOND PAGE
%-------------------------------------------------------------------------------
%\newpage 
%\makebacksidebar 


% \newgeometry{
% 	top=\topbottommargin,
% 	bottom=\topbottommargin,
% 	right=\leftrightmargin,
% 	left=\leftrightmargin
% }

% \cvsection{section}
% \cvsubsection{Subsection}
% \begin{cvtable}
% 	\cvitem{<dates>}{<cv-item title>}{<location>}{<optional: description>}
% \end{cvtable}
\cvsignature

\end{document}
